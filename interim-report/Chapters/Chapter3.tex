\chapter{Project \& Evaluation Plan} 

\label{Chapter3} 
\lhead{Chapter 3. \emph{Project \& Evaluation Plan}} 
\emph{
	This section will not be included in the final report. It includes what I think the key milestones of this project are, and how I should approach implementing them. Because most of this project is new stuff to me, I spent a long time researching it. Writing the background section helped me to understand how I can approach the project, but I do not know how long it will take me to do each part.
}
% 	You should explain what needs to be done in order to complete the project and roughly what you expect the timetable to be. Don’t forget to include the project write-up, as this is a major part of the exercise. It’s important to identify key milestones and also fall-back positions, in case you run out of time.  You should also identify what extensions could be added if time permits.  The plan should be complete and should include those parts that you have already addressed (make it clear how far you have progressed at the time of writing).  This material will not appear in the final report.



\section{Project Key milestones}
\begin{itemize}
	\item Implement \verb+NaClRPCGen+. This is the main deliverable of the project. It will generate JavaScript and C++ header files using an input IDL file. These will be the stubs. This will consist of:
	\begin{itemize}
		\item Finding a suitable WebIDL parser, or making my own.
		\item Use the parser to create NaCl C++ bindings, and create JavaScript stub headers.
		\item Initially, use a simple message format. Then if I have time, use a more efficient format like protobuf.
	\end{itemize}
	\item Implement the `RPCRuntime' in both JavaScript and Native Client. This will be implemented on top of the message passing framework that already exists.
	\item Write an application that uses these tools.
	\item Complete the writeup to show how my RPC framework works, including the architecture and tools used. Give justification for each tool used, noting other alternatives I could have used.
\end{itemize}

\section{Implementation status \& plan}
At the time of writing, none of these things have been implemented, as I spent time to research ideas and think of this plan. I plan to do a simple implementation this term that will include a parser and stub generator for very basic JavaScript and C++ types, e.g. just numbers. This exercise should help me understand how difficult it would be to do it for all types, and it would probably expose some areas I should think about in my architecture and implementation. I could come up with an incremental approach to implement the full product. This would ensure that even if the project is not completed by the deadline, I will at least have it working for \emph{some} types of programs.

\section{Evaluation Plan}
I think evaluation my project will include two types of evaluation, one quantitative and one qualitative.

The quantitative part includes measuring how much overhead the RPC framework adds to the simple message passing approach. I will need to measure this for different types of applications: e.g. applications which need to continuously call RPC functions might behave differently to applications which call them every once in a while.

The qualitative evaluation includes seeing how much development time is saved when using RPC. This could be measured by the number of lines the developer needs to write to achieve the same thing with message passing and with RPC.