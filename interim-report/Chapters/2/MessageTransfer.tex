\section{Data representation}
% This section is incomplete. Points I should mention:

% \begin{itemize}
% 	\item NaCl allows representing data into any form. Examples are XDR or Protocol Buffers.
% 	\item Explain how the message format could increase performance
% 	\item Explain that the changing of message format is done in the client and server side stubs. This mean the message format will need to be supported in both JavaScript and NaCl
% 	\item Examine both XDR and protocol buffers
% \end{itemize}

When designing RPC systems, the data representation of the messages being transferred, including how the parameters are marshalled, needs to be defined. This is because the client and sever might have different architectures that affect how data is represented. There are two types of data representation: implicit typing and explicit typing.

Implicit typing refers to representations which do not encode the names or the types of the parameters when marshalling them. Only the values of the parameters are sent. It is up to the sender and receiver to ensure that the types being sent/received are correct, and this is normally done statically through the IDL files which specify how the message will be structured. 

Explicit typing refers to when the parameter names and types are encoded with the message during marshalling. This increases the size of the messages, but simplifies the process of de-marshalling the parameters.


%---------------------------------------------------------------------------------------------------------------------%
% Implicit message typing
%---------------------------------------------------------------------------------------------------------------------%
\subsection{Implicit Typing}


%---------------------------------------------------------------------------------------------------------------------%
% Explicit message typing
%---------------------------------------------------------------------------------------------------------------------%
\subsection{Explicit Typing}



%% PROTCOL BUFFERS
\subsubsection{Protocol Buffers}
Google Protocol Buffers are ``a language-neutral, platform-neutral, extensible way of serializing structured data for use in communications protocols, data storage, and more" according to the developer guide\footnote{https://developers.google.com/protocol-buffers/docs/overview}. They are used extensively within many Google products, including AppEngine\footnote{Google AppEngine is a Platform as a Service that allows developers to run their applications on the cloud}.

Many RPC implementations which use protocol buffers exist.