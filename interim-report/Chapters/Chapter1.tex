% Chapter Template

\chapter{Introduction} % Main chapter title

\label{Chapter1} % Change X to a consecutive number; for referencing this chapter elsewhere, use \ref{ChapterX}

\lhead{Chapter 1. \emph{Introduction}} % Change X to a consecutive number; this is for the header on each page - perhaps a shortened title
Over the past decades, the web has quickly evolved from being a simple online catalogue of information to becoming a massive distributed platform for web applications that are used by millions of people. Developers have used JavaScript to write web applications that run on the browser, but JavaScript has some limitations. 

One of the problems of JavaScript is performance. JavaScript is a single threaded language with lack of support for concurrency. Although web browser vendors such as Google and Mozilla are continuously improving JavaScript run time performance, it is still a slow interpreted language, especially compared to compiled languages such as C++ (need reference). Many attempts have been made to increase performance of web applications. One of the first solutions was browser plugins that run in the browser, such as Flash or Java Applets. However, these have often created browser bugs and loop-holes that can be used maliciously to compromise security.

Native Client \cite{yee2009native} is a new technology by Google that allows running binary code in a sandboxed environment in the Chrome browser. This new technology will allow web developers to write and use computation-heavy programs that run inside a web application, whilst maintaining the security levels we expect when visiting web applications.

The native code is typically written in C++, though other languages can be supported. The code is compiled and the binary application is sandboxed by verifying the code to ensure no system-calls are made. This is done by compiling the source code by the gcc based NaCl compiler. Because no system calls can be made, the only way an application can communicate with the operating system (for example, to play audio) is through the web browser, which supports several APIs in JavaScript that are secure to use. This means that the fast-performing C++ application needs to communicate with the JavaScript web application.

The way Native Client modules can communicate with the JavaScript web application (and vice versa) is through message passing. This allows for straight forward, asynchronous communication between the native code and the web application. Modern web browsers support the postMessage API which is used for message passing between a web application and one or more web workers\footnote{Web workers\cite{webworkersw3c} are scripts that run in the background of a web page, independent of the web page itself. It is a way of carrying out computations while not blocking the main page's execution. Although they allow concurrency, they are relatively heavy and are not intended to be spawned in large numbers. Typically a web application would have one web worker to carry out computations, and the main page to do most of the view logic (such as click listening, etc.)}. This makes this communication scheme effective.

However, message passing puts more burden on the developer to write the required communication code between the native code and the application. The purpose of this project is to allow developers to easily write Native Client modules by creating a Remote Procedure Call (RPC) framework on top of the existing message passing mechanism. This will allow developers to write Native Client modules in C++ and call the functions directly from JavaScript without having to write the communication code by hand. Similar to how RPC is implemented in SunRPC (need reference), a tool will be provided to compile Interface Definition Language (IDL) files and generate JavaScript and C++ files that can be used as stubs to call the C++ functions directly from JavaScript.
