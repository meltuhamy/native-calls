% Chapter Template

\chapter{Introduction} % Main chapter title

\label{ChapterX} % Change X to a consecutive number; for referencing this chapter elsewhere, use \ref{ChapterX}

\lhead{Chapter X. \emph{Introduction}} % Change X to a consecutive number; this is for the header on each page - perhaps a shortened title

Native Client \cite{yee2009native} is a new technology by Google that allows running binary code in a sandboxed environment in the Chrome browser. This new technology will allow web developers to write and use computation-heavy programs that run inside a web application, whilst mantaining the security levels we expect when visiting web applications.

The native code is typically written in C++, though other languages can be supported. The web application, written in Javascript, can communicate to and from the native code using message passing. This allows for straight forward, asynchronous communication between the native code and the web application. Modern web browsers support the postMessage API which is used for message passing between a web application and one or more web workers. This makes this communication scheme effective.

However, message passing puts more burden on the developer to write the required communication code between the native code and the application. The purpose of this project is to research other methods of communication to see if there is a more effective method that will allow developers to easily call native functions from Javascript. One challenge we immediately face is that the only way communication is supported is through message passing. This means any other communication method, such as remote proceural calls, will need to be implemented on top of the existing message passing framework.

%----------------------------------------------------------------------------------------
%	Background
%----------------------------------------------------------------------------------------


\section{Background}